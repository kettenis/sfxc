\documentclass[11pt]{article}
\usepackage{url}
\usepackage[dvips]{graphicx,rotating,color}
\usepackage{charter}
\title{Translation node web service: \\ Implementation details}
\author{Des Small, \textsc{expr}e\textsc{s} scientific
  programmer, \textsc{jive}}

\begin{document}
\maketitle

\section{Introduction}
This document describes some implementation details of the web services
used to make it possible to run the software correlator on the Grid.

\begin{figure}[h]
\includegraphics[width=1.0\linewidth, angle=0]{old-docs/figures/grid.eps}
\caption{Architecture of Grid-based correlation system}
\end{figure}

Grid\cite{grid} technologies are designed to ``enable the sharing,
selection, and aggregation of a wide variety of geographically
distributed computational resources (such as supercomputers, compute
clusters, storage systems, data sources, instruments, people) and
presents them as a single, unified resource for solving large-scale
compute and data intensive computing applications''.  However, the
model of Grid computing implemented to date does not support
\emph{streaming} data -- the basis of e-\textsc{vlbi} -- but only a
form of batch-mode in which data is transferred to the \textsc{cpu} by
the Grid version of \textsc{ftp} (Grid\textsc{ftp}).

Much of the architectural complexity described in this document is due
to reconciling the streams of data implicit in e-\textsc{vlbi} with the
batch model of Grid computing.  (A lot of the rest is a result of the
'Web Services' framework adopted by the Grid.)  The basic idea is
simple enough, though: as data streams in we simply cut it up into
chunks, and transfer the chunks to the software correlator hosted on
the Grid.


\section{Details}

\subsection{Servers and services}

\begin{description}
  \item[Work-Flow Manager (WFM)]
  \item[VLBI broker] 
  \item[Translation node service]
    (\url{http://huygens:8082/translationnode}).  Hosted at
    \textsc{jive}, this is used by the \textsc{vlbi} broker to request
    data for a specific time range and station.  The data is then cut
    up into chunks and transferred to the Grid\textsc{ftp} service,
    after which the Notification Service is informed of their availability.
  \item[GridFTP server] (\url{melisa.man.poznan.pl/~/})
  \item[Notification Service]  (temporarily at \url{juw32} but intended for 
     \url{http://melisa.man.poznan.pl:8086/vlbiBroker/services/TranslationNodeNotification})
  \item[Mark5 units] (various IP addresses). It is the job of the
    translation node server to find the appropriate Mark5 server given an
    experiment and telescope name.
\end{description}

\subsection{Workflow}

\begin{enumerate}
\item The VLBI broker sends a request (described in
  Table~\ref{tab:transNodeClient}) to the translation node 
  service;
\item The translation node service chops up the data from the Mark5 into
  chunks (on its local file system), and for each chunk the service:
  \begin{enumerate}
    \item Transfers the chunk to the GridFTP server, and 
    \item Sends a message (described in Table~\ref{tab:toNotification}) to
      the notification service. 
  \end{enumerate}
\end{enumerate}

\subsection{SOAP packets}
The contents of the SOAP packet sent from the VLBI broker to the
translation node service is described in
Table~\ref{tab:transNodeClient}.  (Note that fields labelled
-\texttt{IP} and -\texttt{IPAddress} these are in fact generally host
names rather than 4-byte IP addresses.)  All times are given in
so-called Vex format (eg., \texttt{2007y158d18h41m32s}).

The contents of each of the SOAP packets sent from the translation node
service to the notification server is described in
Table~\ref{tab:toNotification}.

Both these packets contain the address of the VLBI broker -- the initiating
client, from our perspective.  Additionally the translation node service
passes its own address to the notification service in the field
\texttt{translationNodeIP}.

Okon\cite{okon} says:
\begin{quotation}
\noindent\texttt{chunkId} - every chunk should have its assigned unique
id number, for easy identification and ordering. So for every new
translation job (experiment), start the counter from value '1' (we
prefer to do not use and skip '0') and just increment the value with
each new chunk. Remember to number the chunks in proper order. This will
make things for VLBI broker a little bit easier.

\vspace{0.5\baselineskip}\noindent\texttt{translationNodeId} - also to help VLBI broker
to instantly distinguish the chunk provider. The IDs will be assigned
to TNs when we will have a complete list of them, right now we can
assume that your TN has an id of '100' (taken from my hat) ;-)
\end{quotation}

Note that \"Ozdemir\cite{oz} implies that there is a one-to-one mapping between
telescopes (ie., Mark5s) and translation nodes, so we should expect to
have multiple translation nodes at \textsc{jive}, and more than one (or
indeed all of them) may run on a single server; this makes it useful to
have a \texttt{translationNodeId} field as well as a
\texttt{translationNodeIP} field.

\begin{table}\label{tab:transNodeClient}
\begin{tabular}{l}
\texttt{TranslationJobRequest} \\ \hline
\texttt{brokerIPAddress} \\
\texttt{chunkSize} (bytes) \\
\texttt{startTime} (in vex format) \\
\texttt{endTime} (in vex format) \\
\texttt{dataLocation} (filename) \\
\texttt{telescopeName} (two-letter code) \\
\texttt{experimentName} (eg. N08C1) 
\end{tabular}
\caption{Packet sent from the VLBI broker to the translation node service}
\end{table}

\begin{table}\label{tab:toNotification}
\begin{tabular}{l}
\texttt{TranslationNodeNotification}\\ \hline
\texttt{brokerIPAddress} \\
\texttt{chunkId} (an integer) \\
\texttt{chunkLocation} (ie., GridFTP server)  \\
\texttt{chunkSize} (bytes)\\
\texttt{startTime} (vex format)\\ 
\texttt{endTime} (vex format)\\
\texttt{translationNodeIP} (ie., sending host) \\
\texttt{translationNodeId} (an integer)
\end{tabular}
\caption{Packet sent from translation node service to notification service} 
\end{table}

\subsection{Omissions}
The transfer to GridFTP servers is not yet implemented in the code.

\begin{thebibliography}{99} 
\bibitem[1]{oz}{\it A short manual on web services},
  H\"usseyin \"Ozdemir (Internal \textsc{jive} report, March, 2008).
\bibitem[2]{okon}{\it Email communication} Marcin Okon, June 2008 
\bibitem[3]{grid}\url{http://www.gridcomputing.com/}
\end{thebibliography}.

\end{document}
